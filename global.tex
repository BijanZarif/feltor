\documentclass[a4paper,12pt]{scrartcl}
\usepackage[utf8x]{inputenc}
\usepackage[T1]{fontenc}

\usepackage{amsmath}
\usepackage{amsfonts}
\usepackage{amssymb}
\usepackage{amsthm}



\usepackage{mathbbol}

\usepackage{graphicx} % Include figure files
\usepackage[margin=10pt,font=small,labelfont=bf,labelsep=colon, format = plain, indention = .5cm]{caption} %schöne Untertitel

\renewcommand{\vec}[1]{{\mathbf{#1}}}
%opening
\title{The nonlocal toefl equations}
\author{Matthias Wiesenberger}

\begin{document}

\maketitle
\begin{abstract}
    The nonlocal form of the toefl equations is presented.
\end{abstract}
\section{Gyrokinetic Hamiltonian}
The first step in a consistent derivation of gyrofluid equations
is to propose a suitably chosen
gyrokinetic action functional, i.e. to introduce all model approximations 
in the fundamental gyrokinetic Lagrangian density already. The equations that
result from a consistent derivation using 
the variational formulation will automatically(!) satisfy 
energy conservation through Noether's theorem. [Sugama, 2000; Brizard, 2000]
That being said, 
we follow [Strintzi and Scott, Pyhsics of Plasmas, 2004] and also reference
[Scott, Physics of Plasmas, 2010]. 
The Hamiltonian is given by
\begin{align}
    H = m\frac{z^2}{2} + wB+eJ_0\phi- \frac{mv_E^2}{2}
    \label{eq:hamiltonian}
\end{align}
where $\vec v_E= (c/B^2)\vec B \times \vec\nabla \phi$. The last term can be 
identified as the 
long wavelength approximation of the gyroscreening potential given in [Scott, 2010].
(Insert $J_0 = 1-\frac{b^2}{4}$ and use $b^2\phi^2 = 2\left(  ( b\phi)^2 + \phi\Delta \phi\right)$).
[Strintzi, 2004] justify that the approximation of $J_0$ only in the 
gyroscreening potential is indeed consistent. We remark that the gyroscreening
term is not a field energy term in this case, contrary to derivations that 
use linearized polarization. 
%We also remark that we follow [Scott, 2010] and do
%not include the electrostatic field energy in the field energy terms of the
%Lagrangian density.
\section{Nonlinear Poisson and the continuity equation}
We omit the derivation and directly state the result of [Strintzi, 2004] for 
the gyrofluid equations. Taken
from their appendix, the polarization equation reads:% in the limit
%of large polarizability
\begin{align}
    \sum_z\left[ \Gamma_{1z} (neZ)_z + 
    \vec \nabla \cdot \frac{(nm)_zc^2}{B^2}\vec\nabla \phi\right] = \frac{1}{4\pi}\Delta \phi
    \label{eq:poisson}
\end{align}
where $\Gamma_{1z} := \left( 1-\frac{\rho_z^2}{2}\Delta \right)^{-1/2}$,
$e$ is the elementary charge and $Z$ is the dimensionless charge of particle
species $z$.
From their equation (43)  we derive ( setting constant temperature, $p = nT$ 
and using the approximation  
$u_{e\parallel} \propto j_{e\parallel} - \nabla_\parallel p 
        = \eta \nabla_\parallel\phi - T_e\nabla_\parallel n_e$
(Ohm's law)
and $u_{z\parallel} =0$):
\begin{subequations}
\begin{align}
    \frac{\partial n_e}{\partial t} + \{\phi, n_e\} &= c(e \phi - T_e n_e) + n_e\kappa(\phi) -
    \frac{T_e}{e}\kappa(n_e)  \\
    \frac{\partial n_z}{\partial t} + \{\psi_z, n_z\} &= n_z\kappa(\psi_z) +
    \frac{T_z}{eZ}\kappa(n_z) 
    \label{}
\end{align}
\end{subequations}
where $\psi_z = \Gamma_{1z} \phi$. Following [Scott, private correspondance] we 
 neglect $\Omega_z$ in the limit of vanishing temperature variations.
\section{dimensional analysis}
We use 
\begin{subequations}
    \begin{align}
        x = \rho_s x', \ \ \ \ \  v = c_s v', \ \ \ \ \ 
            &\partial_t = \Omega \partial_t', \\
        \phi = \frac{T_e}{e}\phi',\ \ \ \ \
        &n_z  = n_{z0} n_z'%\frac{1}{\rho_s^3}n_z'
        \label{}
    \end{align}
    \label{}
\end{subequations}with 
 ion gyroradius at electron temperature $\rho_s = \frac{c\sqrt{m_i T_e}}{eB}$,
 ion gyrofrequency $\Omega = \frac{eB}{cm_i}$ and
the ion sound speed at electron temperature $c_s = \sqrt{\frac{T_e}{m_i}}$. 
Note that $c_s = \Omega \rho_s$
and $\frac{B}{c}\Omega \rho_s^2 = \frac{T_e}{e}$.
Also $n_z'$ now should be of order one. 
Furthermore we write
\begin{subequations}
    \begin{align}
        \mu_z = \frac{m_z}{Zm_i}\\
        \tau_z = \frac{T_z}{ZT_e} \\
        a_z = \frac{Zn_{z0}}{n_{e0}}
        \label{}
    \end{align}
    \label{}
\end{subequations}
With this choice Poisson's equation reads (omitting ')
\begin{align}
    \sum_z \left[ a_z \Gamma_{1z}n_z + a_z\mu_z\nabla\cdot(n_z \nabla\phi)\right] = \frac{1}{4\pi}\left(  \frac{\lambda_D}{\rho_s} \right)^2 \Delta \phi 
    \label{}
\end{align}
with $\Gamma_{1z} = \left(1-\frac{1}{2}\tau_z\mu_z \Delta\right)^{-1}$ and the 
Debye length $\lambda_D^2 = \frac{T_e}{n_{e0}e^2}$. We note 
that $\lambda_D \ll \rho_s$ and neglect the term on the right side. (Q: is this
actually a condition for the up-to-now free parameter $n_{e0}$?)

The dimension of the curvature operator is found to be
$\kappa = \frac{\Omega e}{T_e}\kappa'$, 
so that the dimension of $\kappa(\phi)$ is $\Omega$.
Thus the continuity equations (omitting the ') read
\begin{subequations}
\begin{align}
    \frac{\partial n_e}{\partial t} + \{\phi, n_e\} &= d( \phi - n_e) + n_e\kappa(\phi) -
    \kappa(n_e) \\
    \frac{\partial n_z}{\partial t} + \{\psi_z, n_z\} &= n_z\kappa(\psi_z) +
    \tau_z\kappa(n_z) 
    \label{}
\end{align}
\end{subequations}
If we separate one single charged ion species from the impurity species we 
have to solve the system
\begin{subequations}
    \begin{align}
    \frac{\partial n_e}{\partial t} + \{\phi, n_e\} &= d( \phi - n_e) + n_e\kappa(\phi) -
    \kappa(n_e) \\
    \frac{\partial n_{i/z}}{\partial t} + \{\Gamma_{1i/z}\phi, n_{i/z}\} &= 
    n_{i/z}\kappa(\Gamma_{1i/z}\phi) + \tau_{i/z}\kappa(n_{i/z}) \\
    %\frac{\partial n_z}{\partial t} + \{\Gamma_{1z}\phi, n_z\} &= n_z\kappa(\Gamma_{1z}\phi) +
    %\tau_z\kappa(n_z) \\
    \vec \nabla\cdot\left[ (\mu_i n_i + a_z \mu_z\ n_z)\vec \nabla \phi \right] &= 
        n_e - \Gamma_{1i}n_i + a_z \Gamma_{1z} n_z
        \label{}
    \end{align}
    \label{}
\end{subequations}

\section{Boundary conditions}
We follow [Garcia et al., Phys. Scr. T122 89, 2006] in the choice of 
the boundary conditions. 
In the $y$ coordinate all quantities are periodic. 
In the $x$ coordinate we distinguish between the inner wall situated at $x=0$ and
the outer wall situated at $x=L_x$.
We impede all particle fluxes on the inner boundary by setting
\begin{align}
    \phi = 0, \ \ \frac{\partial n_z}{\partial x} = 0
    \label{}
\end{align} 
and model the incoming particle flux as a source term $S\propto exp\left( -\frac{x^2}{2\sigma_x^2} \right)$, i.e. a half gaussian situated at the edge. 
On the other side we choose
\begin{align}
    \phi = 0, \ \ n_z-1=0
    \label{}
\end{align}
i.e. we stop the convective flux accross the boundaries but allow for a diffusive
flux. (A diffusive term is still absent in our equations, note 
that $\kappa = \kappa_y\partial_y$, i.e. curvature driven flux is in $y$ only)
The SOL is modelled via a damping term in the density equations. (Q: What happens
with the coupling term $d(\phi-n_e)$ in HW equations and what is 
with the modified HW and zonal flows?) If the last closed flux surface is situated
at $x_l$ the particle density is damped towards $n_0$ in the whole region
next to the SOL. $\delta \propto [1+\tanh(\frac{x-x_l}{\sigma_l}]$. This
models particle losses due to open field lines.


\end{document}

