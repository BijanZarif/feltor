\documentclass[a4paper,12pt]{scrartcl}
\usepackage[utf8x]{inputenc}
\usepackage[T1]{fontenc}

\usepackage{amsmath}
\usepackage{amsfonts}
\usepackage{amssymb}
\usepackage{amsthm}



\usepackage{mathbbol}

\usepackage{graphicx} % Include figure files
\usepackage[margin=10pt,font=small,labelfont=bf,labelsep=colon, format = plain, indention = .5cm]{caption} %schöne Untertitel

\renewcommand{\vec}[1]{{\mathbf{#1}}}
%opening
\title{Report}
\author{Matthias  Wiesenberger}

\begin{document}

\maketitle

\begin{abstract}
    Revision of dimensionlesss toefl equations for implicit karniadakis scheme.
\end{abstract}

\section{The Toefl Equations}

In order to use the Karniadakis scheme our equations have to be of the form 
\begin{align}
 \partial_t \vec v = \vec N(\vec v) + \vec L( \vec v)
\end{align}
where $\vec v = (n_e, n_i, n_z)$, $\vec N$ symbolizes the nonlinear part and $\vec L$ the linear part. 
The toefl equations are: (A hat above a quantity symbolizes an operator)
\begin{subequations}
\begin{align}
    \partial_t n_e &= -\{\phi,n_e\} +\hat\kappa(\phi - n_e) + d(\phi - n_e) + g_e\partial_y\phi - \nu\Delta^2n_e \\
    \partial_t n_i &= -\{\hat\Gamma_{1i}\phi,n_i\} +\hat\kappa(\hat\Gamma_{1i}\phi +\tau_i n_i) + g_i\partial_y\hat\Gamma_{1i}\phi - \nu\Delta^2n_i \\
    \partial_t n_z &= -\{\hat\Gamma_{1z}\phi,n_z\} +\hat\kappa(\hat\Gamma_{1z}\phi +\tau_z n_z) + g_z\partial_y\hat\Gamma_{1z}\phi - \nu\Delta^2n_z
\end{align}
\label{eq:toefl_equations}
\end{subequations}
In our equations each species is specified by the dimensionless temperature $\tau_s$, the
dimensionless charge $a_s$ and the dimensionless mass $\mu_s$. Each species
has a background gradient $g_s$. Note that 
$\tau_e = a_e = -1$ and $\mu_e = 0$. $\nu$ is a artificial numerical viscosity.
The nonlinear part consists of the Arakawa brackets
\begin{align}
    N_e(\vec v) &= \{n_e,\ \phi\}\\
    N_i(\vec v) &= \{n_i,\ \hat\Gamma_{1i}\phi\}\\
    N_z(\vec v) &= \{n_z,\ \hat\Gamma_{1z}\phi\}
\end{align}
where $\phi = \phi(\vec v)$ is given by the local (linear) Poisson equation
\begin{align}
    \phi = \frac{1}{\hat\rho}(n_e -a_i\hat\Gamma_{1i}n_i -a_z\hat\Gamma_{1z}n_z) =: \hat\phi_e n_e + \hat\phi_i n_i + \hat\phi_z n_z
\end{align}
$\hat\rho$ and $\hat\Gamma_1$ are written as 
\begin{align}
    \hat\rho &= \frac{a_i\mu_i \Delta}{1-\tau_i\mu_i\Delta} + \frac{a_z\mu_z\Delta}{1-\tau_z\mu_z\Delta}\\
    \hat\Gamma_{1i} &= [1-0.5\tau_i\mu_i\Delta]^{-1}\\
    \hat\Gamma_{1z} &= [1-0.5\tau_z\mu_z\Delta]^{-1}\\
    %\hat\Gamma_{2i} &= \frac{1}{2} \tau_i \mu_i \Delta \hat\Gamma_{1i}^2\\
    %\hat\Gamma_{2z} &= \frac{1}{2} \tau_z \mu_z \Delta \hat\Gamma_{1z}^2\\
    \Delta          &= \partial_x^2 + \partial_y^2
\end{align}
which in fourier space reduce to (real) coefficients (dependent on the set of expansion functions used).
The linear part of our equations can be written in matrix form as
\begin{align}
    \begin{pmatrix}
        \hat L_e ( n_e, n_i, n_z) \\
        \hat L_i ( n_e, n_i, n_z) \\
        \hat L_z ( n_e, n_i, n_z)
    \end{pmatrix} = 
    \begin{pmatrix}
        \hat P_e\hat \phi_e - \hat\kappa -d -\nu\Delta^2 & \hat P_e\hat \phi_i &\hat P_e \hat \phi_z\\
        \hat P_i\hat \phi_e & \hat P_i \hat \phi_i + \hat\kappa \tau_i - \nu\Delta^2 & \hat P_i \hat \phi_z \\
        \hat P_z\hat \phi_e & \hat P_z \hat \phi_i & \hat P_z \hat \phi_z + \hat\kappa \tau_z - \nu\Delta^2
    \end{pmatrix}
    \begin{pmatrix}
        n_e \\n_i\\n_z
    \end{pmatrix}
    \label{eq:linear_part}
\end{align}
with 
\begin{align}
    \hat \kappa &= \kappa_y \partial_y \\
    \hat P_e &= \hat\kappa + d +g_e\partial_y \\
    \hat P_i &= \hat\kappa \hat\Gamma_{1i}  + g_i \partial_y \hat\Gamma_{1i}\\
    \hat P_z &= \hat\kappa \hat\Gamma_{1z} + g_z \partial_y\hat\Gamma_{1z}
\end{align}
Note that the operators $\hat P_s$ are the prefactors of $\phi$ in equation (\ref{eq:toefl_equations}).

The Karniadakis scheme now consists of the following two steps:
\begin{align}
    \vec {\hat v}^n &= \sum_{q=0}^2 \alpha_q\vec v^{n-q} + \Delta t(\sum_{q=0}^2\beta_q \vec N(\vec v^{n-q})) \ &\text{in x-space} \\
    (\gamma_0 \vec \Eins  - \Delta t \vec L) \vec v^{n+1} &= \vec {\hat v}^n  \ &\text{in k-space}
\end{align}
with coefficients
\begin{align}
    \alpha_0 &= 3 \  &&\beta_0 = 3 \nonumber \\
    \alpha_1 &= -\frac{3}{2} && \beta_1 = -3 && \gamma_0 = \frac{11}{6}\label{eq:coefficients}\\
    \alpha_2 &= \frac{1}{3} && \beta_2 = 1 \nonumber
\end{align}
That means we first have to compute the nonlinearity $\vec N(\vec v^{n})$ and add everything up to $\vec {\hat v}^n$. The second step consists of a fourier transformation of $\vec v^n_{temp}$ and the inversion 
of $(\gamma_0 - \Delta t \vec L)$ in fourier space. Note that this is a simple 
$(3\times3)$ matrix inversion as long as our base functions are eigenvectors of $\vec L$. (cf. chapter \ref{sec:Fourier}) Depending on the boundary conditions we have to choose different base functions.
Note that from $n^{n+1}_e$, $n^{n+1}_i$ and $n^{n+1}_z$ in fourier space we also have to compute $\phi^{n+1}$, $\Gamma_{1i}\phi^{n+1}$ and $\Gamma_{1z}\phi^{n+1}$ for the arakawa bracket of the next step.
Also note that by choosing other values for the coefficients (\ref{eq:coefficients}) 
we can end up with an Euler or a 2nd order scheme which is important for the 
initialisation of the Karniadakis scheme.

\section{ Boundary conditions} 
The quantities $n_e$, $n_i$, $n_z$ and $\phi$ are (small) fluctuating quantities on top of
some (fixed) background gradients. Since the topology of our simulation box is periodic in the $y$-direction 
we choose periodic boundary conditions in $y$. 
Periodc boundary conditions in $x$ are sufficient as long as the gradient length scale is large 
compared to the simulation box length $l_x$.
Other boundary conditions are those where the fluctuating quantities 
$n_e$, $n_i$ and $n_z$ vanish at $z=0$ and $z=l_x$. 
When discussing boundary conditions for $\phi$, take into account that the actually measurable 
quantity is the electric field $\vec E$ respectively the $\vec E \times \vec B$ velocity $\vec v_E$.
We require $v_x $ respectively $E_y = \partial_y\phi$ to vanish at the boundary, which means that the 
boundary is impermeable for the fluid. With the gauge freedom for the potential $\phi$
we translate this into the condition $\phi( x=0,l_x, y) = 0$. Since our simulation box is
inside the plasma we require that the plasma streams freely along the boundary wall.
(Free slip boundary conditions)
This means that the gradient $\partial_x v_y = \partial_x^2 \phi$ has to vanish (no fricition). 
All in all we require
\begin{subequations}
\begin{align}
    n_s( x=0, y,t) = n_s( x=l_x, y,t) = 0 \\
    \phi( x=0, y,t) = \phi( x=l_x, y,t) = 0\\
    \partial_x^2\phi( x=0, y,t) = \partial_x^2\phi( x=l_x, y,t) = 0\\
    n_s( x, y+l_y, t) = n_s( x, y,t) \\
    \phi( x, y+l_y, t) = \phi( x, y,t)
\end{align}
\label{eq:boundary_conditions}
\end{subequations}
\section{Fourier transformations}
\label{sec:Fourier}
We know that the base functions $e^{iky}$ are eigenfunctions of every linear
operator with complex eigenvalues. That means that when we transform Equation \ref{eq:linear_part} in $y$ we simply replace $\partial_y$ by $ik$ and $\partial_y^2$ by $-k^2$. 
In the $x$-direction a sine transformation of type I or II is the method of choice 
to naturally fulfill the Dirichlet boundary conditions. Note that sine functions $\sin(k x)$ are 
eigenfunction of $\partial_x^2$ so all operators are diagonal.


%\begin{figure}[htbp]
% \centering
% \includegraphics[width = 0.5\textwidth]{cos.eps}
% \caption[optional für Bildreferenz]
%  {Untertext blalslslslsl} 
%\label{Abb:Konvektionszelle_Teilchen}
%\end{figure}
\end{document}
